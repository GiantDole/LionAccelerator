\section{Resources \& Opportunities}
Our network of universities and blockchain clubs allow us to accumulate the best skills and value in one event. The question is not whether the resources are available, but how to combine them in the most efficient way by working hand in hand towards the same goal. We have created an event scheme that intends to fulfill the needs of all participating parties, as well as incentivizing participation by any means. \\
The following three parties are the most crucial to a successful event: \textbf{Blockchain clubs \& Universites}, \textbf{Sponsors}, and \textbf{Funders}. In the following we describe the role of each party, what resources they contribute, and how this can be merged to achieve a frictionless and feature-rich event, benefiting all parties:
%
%
\paragraph{Blockchain clubs \& Universites} provide the actual value for this event. This network will gather the best rising developers who are either members of blockchain clubs, or pursuing a blockchain class at their university with a smart contract project or similar. Solely at Columbia there exists a blockchain class with about 40 projects being done by May. We will reach out to all classes and blockchain clubs to apply to our event with their projects. The benefit of the best upcoming projects to participate in this event is because in the event of success they will receive a fair funding and support they wouldn't usually or easily find in the open market. Another factor why this is true is because these projects have been determined to be the best ones and their security and functionality has been secured through rigorous and incentivized peer-review. Therefore, funders have a significantly increased assurance of added value.\\
Besides the value of contributing projects, this community is full of useful skills. As we intend to power this event by a blockchain, we require smart developers to help us with this plan. Therefore, we will set up an incentive system rewarding developers and students for their contribution. In this way there is no need to employ contributors but pay the best students in the country for fractioned work they deliver. Read more about how this incentive system will work in \autoref{sec:Incentive System}.

\paragraph{Sponsors} allow the whole planning process to actually take off. They provide us with initial funding to develop the framework of the event and monetary resources for the incentive system. As the whole incentive system and payments are based on a blockchain, the resource usage will be easy to monitor for our sponsors. Furthermore, these sponsors will have advantages in the event itself which are described in more detail in \autoref{subsec:Incentive_sponsors}. We are looking to get two main sponsors on board:
\begin{itemize}
    \item \textbf{Blockchain sponsor}: will power the event with their blockchain. The whole web3 event will base on transactions on their blockchain and showcase their product in an elite setting. Additionally, all the smart contracts that will be written for the event on their blockchain might result in actual dApps after the event. We also hope to receive support to make our web3 applications secure, especially such that our incentive system can't be exploited. Such a sponsor will be labels as \textit{powered by} under our event logo.
    \item \textbf{Event sponsor}: will be the main monetary provider for our event. Their logo will be omnipresent in all invitations, messages, website, and the event itself. The event sponsor has secured themselves a position to participate in the event process with conditions to be determined. They could take part in the application process, selection process, speak on the event, provide workshops on the event, or simply secure themselves a position for funding the best team in some track of their choice. Regarding the application process this also provides them with the benefit of having insights to all the offers made by funders and what the focus on accepted projects are. The size of our event and how many skills we can gather to make this vision happen mainly depends on this sponsor.
\end{itemize}
In addition to these main sponsors we have developed three different classes of sponsors: \textbf{gold sponsors}, \textbf{silver sponsors}, and \textbf{bronze sponsors}. They will provide additional funding in return for predetermined advantages when the event takes place as described in \autoref{subsec:Incentive_sponsors}. In later rounds the funders will be determined, who will not have these exclusive advantages anymore.

\paragraph{Funders} consist of sponsors and investors joining in at a later point. The funders are probably the true beneficiaries of this event: They get to fund the best upcoming projects developed by the smartest students/developers around the globe. They can rest assured that their investment will create actual value because the projects underwent an application process and selection phase they can take part in. Additionally, the skills one would usually need to assess a project will be provided by the teams themselves through extensive, rigorous, and incentivized peer-review.\\
While the funding they provide is less constrained than usual, the increased value should make up for it to build a trade-off satisfying both parties. During the investor registration phase, funders will register with the funding they want to provide, optional additional funding they would provide if an idea really convinces them and satisfies their need, as well as the constraints attached to their funding. We will then select the "best" funding opportunities for the teams and incorporate them into the event as described in \autoref{subsec:selection}.
